\documentclass[]{article}
\usepackage{lmodern}
\usepackage{amssymb,amsmath}
\usepackage{ifxetex,ifluatex}
\usepackage{fixltx2e} % provides \textsubscript
\ifnum 0\ifxetex 1\fi\ifluatex 1\fi=0 % if pdftex
  \usepackage[T1]{fontenc}
  \usepackage[utf8]{inputenc}
\else % if luatex or xelatex
  \ifxetex
    \usepackage{mathspec}
  \else
    \usepackage{fontspec}
  \fi
  \defaultfontfeatures{Ligatures=TeX,Scale=MatchLowercase}
\fi
% use upquote if available, for straight quotes in verbatim environments
\IfFileExists{upquote.sty}{\usepackage{upquote}}{}
% use microtype if available
\IfFileExists{microtype.sty}{%
\usepackage{microtype}
\UseMicrotypeSet[protrusion]{basicmath} % disable protrusion for tt fonts
}{}
\usepackage[margin=1in]{geometry}
\usepackage{hyperref}
\hypersetup{unicode=true,
            pdftitle={Study of census attributes in Bandundu},
            pdfborder={0 0 0},
            breaklinks=true}
\urlstyle{same}  % don't use monospace font for urls
\usepackage{color}
\usepackage{fancyvrb}
\newcommand{\VerbBar}{|}
\newcommand{\VERB}{\Verb[commandchars=\\\{\}]}
\DefineVerbatimEnvironment{Highlighting}{Verbatim}{commandchars=\\\{\}}
% Add ',fontsize=\small' for more characters per line
\usepackage{framed}
\definecolor{shadecolor}{RGB}{248,248,248}
\newenvironment{Shaded}{\begin{snugshade}}{\end{snugshade}}
\newcommand{\KeywordTok}[1]{\textcolor[rgb]{0.13,0.29,0.53}{\textbf{#1}}}
\newcommand{\DataTypeTok}[1]{\textcolor[rgb]{0.13,0.29,0.53}{#1}}
\newcommand{\DecValTok}[1]{\textcolor[rgb]{0.00,0.00,0.81}{#1}}
\newcommand{\BaseNTok}[1]{\textcolor[rgb]{0.00,0.00,0.81}{#1}}
\newcommand{\FloatTok}[1]{\textcolor[rgb]{0.00,0.00,0.81}{#1}}
\newcommand{\ConstantTok}[1]{\textcolor[rgb]{0.00,0.00,0.00}{#1}}
\newcommand{\CharTok}[1]{\textcolor[rgb]{0.31,0.60,0.02}{#1}}
\newcommand{\SpecialCharTok}[1]{\textcolor[rgb]{0.00,0.00,0.00}{#1}}
\newcommand{\StringTok}[1]{\textcolor[rgb]{0.31,0.60,0.02}{#1}}
\newcommand{\VerbatimStringTok}[1]{\textcolor[rgb]{0.31,0.60,0.02}{#1}}
\newcommand{\SpecialStringTok}[1]{\textcolor[rgb]{0.31,0.60,0.02}{#1}}
\newcommand{\ImportTok}[1]{#1}
\newcommand{\CommentTok}[1]{\textcolor[rgb]{0.56,0.35,0.01}{\textit{#1}}}
\newcommand{\DocumentationTok}[1]{\textcolor[rgb]{0.56,0.35,0.01}{\textbf{\textit{#1}}}}
\newcommand{\AnnotationTok}[1]{\textcolor[rgb]{0.56,0.35,0.01}{\textbf{\textit{#1}}}}
\newcommand{\CommentVarTok}[1]{\textcolor[rgb]{0.56,0.35,0.01}{\textbf{\textit{#1}}}}
\newcommand{\OtherTok}[1]{\textcolor[rgb]{0.56,0.35,0.01}{#1}}
\newcommand{\FunctionTok}[1]{\textcolor[rgb]{0.00,0.00,0.00}{#1}}
\newcommand{\VariableTok}[1]{\textcolor[rgb]{0.00,0.00,0.00}{#1}}
\newcommand{\ControlFlowTok}[1]{\textcolor[rgb]{0.13,0.29,0.53}{\textbf{#1}}}
\newcommand{\OperatorTok}[1]{\textcolor[rgb]{0.81,0.36,0.00}{\textbf{#1}}}
\newcommand{\BuiltInTok}[1]{#1}
\newcommand{\ExtensionTok}[1]{#1}
\newcommand{\PreprocessorTok}[1]{\textcolor[rgb]{0.56,0.35,0.01}{\textit{#1}}}
\newcommand{\AttributeTok}[1]{\textcolor[rgb]{0.77,0.63,0.00}{#1}}
\newcommand{\RegionMarkerTok}[1]{#1}
\newcommand{\InformationTok}[1]{\textcolor[rgb]{0.56,0.35,0.01}{\textbf{\textit{#1}}}}
\newcommand{\WarningTok}[1]{\textcolor[rgb]{0.56,0.35,0.01}{\textbf{\textit{#1}}}}
\newcommand{\AlertTok}[1]{\textcolor[rgb]{0.94,0.16,0.16}{#1}}
\newcommand{\ErrorTok}[1]{\textcolor[rgb]{0.64,0.00,0.00}{\textbf{#1}}}
\newcommand{\NormalTok}[1]{#1}
\usepackage{longtable,booktabs}
\usepackage{graphicx,grffile}
\makeatletter
\def\maxwidth{\ifdim\Gin@nat@width>\linewidth\linewidth\else\Gin@nat@width\fi}
\def\maxheight{\ifdim\Gin@nat@height>\textheight\textheight\else\Gin@nat@height\fi}
\makeatother
% Scale images if necessary, so that they will not overflow the page
% margins by default, and it is still possible to overwrite the defaults
% using explicit options in \includegraphics[width, height, ...]{}
\setkeys{Gin}{width=\maxwidth,height=\maxheight,keepaspectratio}
\IfFileExists{parskip.sty}{%
\usepackage{parskip}
}{% else
\setlength{\parindent}{0pt}
\setlength{\parskip}{6pt plus 2pt minus 1pt}
}
\setlength{\emergencystretch}{3em}  % prevent overfull lines
\providecommand{\tightlist}{%
  \setlength{\itemsep}{0pt}\setlength{\parskip}{0pt}}
\setcounter{secnumdepth}{0}
% Redefines (sub)paragraphs to behave more like sections
\ifx\paragraph\undefined\else
\let\oldparagraph\paragraph
\renewcommand{\paragraph}[1]{\oldparagraph{#1}\mbox{}}
\fi
\ifx\subparagraph\undefined\else
\let\oldsubparagraph\subparagraph
\renewcommand{\subparagraph}[1]{\oldsubparagraph{#1}\mbox{}}
\fi

%%% Use protect on footnotes to avoid problems with footnotes in titles
\let\rmarkdownfootnote\footnote%
\def\footnote{\protect\rmarkdownfootnote}

%%% Change title format to be more compact
\usepackage{titling}

% Create subtitle command for use in maketitle
\newcommand{\subtitle}[1]{
  \posttitle{
    \begin{center}\large#1\end{center}
    }
}

\setlength{\droptitle}{-2em}

  \title{Study of census attributes in Bandundu}
    \pretitle{\vspace{\droptitle}\centering\huge}
  \posttitle{\par}
    \author{}
    \preauthor{}\postauthor{}
    \date{}
    \predate{}\postdate{}
  

\begin{document}
\maketitle

\subsection{Understanding incomplete
survey}\label{understanding-incomplete-survey}

\subsubsection{Section 1: Overview}\label{section-1-overview}

We study here the attributes of census observation in Bandundu that are
reported as either \texttt{residential} or \texttt{mixte}. Our first
concern goes to status of each observation, namely if it is completed or
not. To do so let's understand what is under variable

\begin{longtable}[]{@{}lr@{}}
\caption{Survey status for each observation}\tabularnewline
\toprule
Status & Count\tabularnewline
\midrule
\endfirsthead
\toprule
Status & Count\tabularnewline
\midrule
\endhead
Complete & 20714\tabularnewline
Incomplete & 217\tabularnewline
\bottomrule
\end{longtable}

\begin{verbatim}
## Linking to GEOS 3.6.1, GDAL 2.2.3, proj.4 4.9.3
\end{verbatim}

\includegraphics{atttributes_study_files/figure-latex/unnamed-chunk-4-1.pdf}

\subsubsection{Section 2: Sources of false
incompleteness}\label{section-2-sources-of-false-incompleteness}

\paragraph{\texorpdfstring{1. False report in
\texttt{Survey\_status}}{1. False report in Survey\_status}}\label{false-report-in-survey_status}

\emph{Situation}: Here we have on observation reported as
\texttt{incomplete} but with household information filled and ``nothing
to report'' on the \texttt{Comment} section.

\emph{Solution}: We change it to \texttt{complete}. Concerns \textbf{1}
observation.

\begin{longtable}[]{@{}lllrll@{}}
\toprule
cluster\_id & building\_type & number\_of\_household & population &
comments & status\tabularnewline
\midrule
\endhead
drc\_maindombe\_0062 & Residential & 1 & 14 & Ras &
Incomplete\tabularnewline
\bottomrule
\end{longtable}

\paragraph{2. Irrelevant buildings}\label{irrelevant-buildings}

\emph{Situation}: as shown in the \texttt{comments}, some buildings
reported as incomplete are actually not building that should contains
people. Example: buildings reported abandonned, toilets, non occupied
etc.

\emph{Solution}: we remove them based on the pattern in the
\texttt{Comments} section. Concerns \textbf{94} observations.

\begin{Shaded}
\begin{Highlighting}[]
\NormalTok{pattern =}\StringTok{ }\KeywordTok{c}\NormalTok{(}\StringTok{'on occup'}\NormalTok{, }\StringTok{'Abandonn'}\NormalTok{, }\StringTok{'vide'}\NormalTok{, }\StringTok{'noccup'}\NormalTok{,}
            \StringTok{"existe plus"}\NormalTok{,}\StringTok{'En construction'}\NormalTok{,}\StringTok{'Boutique'}\NormalTok{,}\StringTok{'habit'}\NormalTok{,}
            \StringTok{'Douche'}\NormalTok{, }\StringTok{'Wc'}\NormalTok{, }\StringTok{'detrui'}\NormalTok{, }\StringTok{'toilette'}\NormalTok{, }\StringTok{'Non occup'}\NormalTok{,}
            \StringTok{'Cuisine'}\NormalTok{, }\StringTok{'Abondonnee'}\NormalTok{,}\StringTok{'Deplaement'}\NormalTok{, }\StringTok{'occup'}\NormalTok{,}\StringTok{'Inau'}\NormalTok{)}
\NormalTok{census_res =census_res }\OperatorTok\StringTok{ }\KeywordTok{filter}\NormalTok{(}\OperatorTok{!}\NormalTok{(status}\OperatorTok{==}\StringTok{'Incomplete'} \OperatorTok{&}\StringTok{ }\KeywordTok{grepl}\NormalTok{(}\KeywordTok{paste}\NormalTok{(pattern, }\DataTypeTok{collapse =} \StringTok{"|"}\NormalTok{),comments)))}
\end{Highlighting}
\end{Shaded}

\paragraph{3. Duplicate of completed
observations}\label{duplicate-of-completed-observations}

\emph{Situation}: Sometimes when the surveyor came back to the
household, he/she didn't reopen the previous incomplete form such that
for the same building we got two ids and two surveys, one completed and
one incompleted (the first one in time). To flag it we based our
analysis on the GPS coordinates (5 digits accuracy for the longitude, 6
for the lattitude), two buildings having the same coordinates were
considered as the same.

\emph{Solution}: we remove these incomplete observations. Concerns
\textbf{9} observations

\begin{quote}
Caveat: I'm not sure that this procedure is correct since in the full
data table, there are 272 entries that have non-unique GPS coordinates.
\end{quote}

\begin{longtable}[]{@{}lr@{}}
\caption{Survey status for each observation after
modifications}\tabularnewline
\toprule
Status & Count\tabularnewline
\midrule
\endfirsthead
\toprule
Status & Count\tabularnewline
\midrule
\endhead
Complete & 20714\tabularnewline
Incomplete & 105\tabularnewline
{[}{]}(atttribut &
es\_study\_files/figure-latex/unnamed-chunk-9-1.pdf)\tabularnewline
\bottomrule
\end{longtable}

\subsubsection{Section 3 Caracteristics of incomplete
survey}\label{section-3-caracteristics-of-incomplete-survey}

\paragraph{Brief description}\label{brief-description}

For the remaining incomplete observations: we have information about: -
the number of floors (all at 1 except one at 174 considered as
irrelevant outlier) - the number of household in the building (all 1
except one at 2)

but no more. We have no other information for imputation.

\subsection{Analysis of completed
survey}\label{analysis-of-completed-survey}

\subsubsection{Additional buildings
surveyed}\label{additional-buildings-surveyed}

when we look at the \texttt{household} dataset, we notice that 30
buildings have been surveyed without being in the \texttt{building}
dataset. They were reported as \texttt{non-residential} (even though
they are absent from the \texttt{building} dataset corresponding to the
non-residential ones).

What does it mean: we have more observations and we need to merge them
with the others. It's possible because they have a building\_id and a
GPS coordinates. But we lacked the \texttt{comments}, the
\texttt{interviewer\_name}and the \texttt{number\_of\_floors} that are
the information reported only at the building level.

\begin{verbatim}
## [1] "Number of additional observations: 30"
\end{verbatim}

\begin{verbatim}
## [1] "Final number of buildings surveyed: 20849"
\end{verbatim}

\subsubsection{Flag incompleteness at household
level}\label{flag-incompleteness-at-household-level}

We know the number of households that are supposed to live in the
building from the \texttt{building} dataset and we know the number of
households that have actually have been interviewed from the
\texttt{household} dataset. We can thus flag buildings where not every
household has been interviewed, information of prime importation for
imputation purposes.

\emph{NB}: \texttt{calculated\_no\_household} and
\texttt{number\_of\_household} differs only for one observation that is

\begin{longtable}[]{@{}lllllrrr@{}}
\toprule
cluster\_id & building\_type & number\_of\_floors &
calculated\_no\_household & comments & population & hh\_obs &
nb\_hh\_incomp\tabularnewline
\midrule
\endhead
drc\_maindombe\_0005 & Residential & 1 & 3 & Ras & 9 & 2 &
1\tabularnewline
drc\_kwilu\_0075 & Residential & 1 & 3 & Erreur de ddoublement & 4 & 1 &
2\tabularnewline
drc\_maindombe\_0062 & Residential & 1 & 2 & Ras & 14 & 1 &
1\tabularnewline
drc\_maindombe\_0079 & Residential & 1 & 2 & Ras & 7 & 1 &
1\tabularnewline
drc\_maindombe\_0079 & Residential & 1 & 2 & Ras & 5 & 1 &
1\tabularnewline
drc\_kwilu\_0075 & Residential & 1 & 3 & Ras & 4 & 1 & 2\tabularnewline
drc\_maindombe\_0050 & Residential & 1 & 2 & /Refus ( 2) & 1 & 1 &
1\tabularnewline
drc\_maindombe\_0059 & Residential & 1 & 35 & Ras & 1 & 1 &
34\tabularnewline
drc\_maindombe\_0026 & Residential & 1 & 2 & Absence de personne mousse
donner les informations & 2 & 1 & 1\tabularnewline
drc\_kwilu\_0045 & Residential & 1 & 2 & Ras & 1 & 1 & 1\tabularnewline
drc\_maindombe\_0028 & Residential & 1 & 2 & Mnage et cuisine & 1 & 1 &
1\tabularnewline
drc\_maindombe\_0059 & Mixed & 1 & 2 & Deuxieme piece est une pallote &
10 & 1 & 1\tabularnewline
drc\_kwilu\_0075 & Residential & 1 & 2 & Erreur de ddoublement & 3 & 1 &
1\tabularnewline
\bottomrule
\end{longtable}

\begin{verbatim}
## [1] "Potential missing household observation: 13"
\end{verbatim}


\end{document}
